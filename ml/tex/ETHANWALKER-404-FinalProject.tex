
% Default to the notebook output style

    


% Inherit from the specified cell style.




    
\documentclass[11pt]{article}

    
    
    \usepackage[T1]{fontenc}
    % Nicer default font (+ math font) than Computer Modern for most use cases
    \usepackage{mathpazo}

    % Basic figure setup, for now with no caption control since it's done
    % automatically by Pandoc (which extracts ![](path) syntax from Markdown).
    \usepackage{graphicx}
    % We will generate all images so they have a width \maxwidth. This means
    % that they will get their normal width if they fit onto the page, but
    % are scaled down if they would overflow the margins.
    \makeatletter
    \def\maxwidth{\ifdim\Gin@nat@width>\linewidth\linewidth
    \else\Gin@nat@width\fi}
    \makeatother
    \let\Oldincludegraphics\includegraphics
    % Set max figure width to be 80% of text width, for now hardcoded.
    \renewcommand{\includegraphics}[1]{\Oldincludegraphics[width=.8\maxwidth]{#1}}
    % Ensure that by default, figures have no caption (until we provide a
    % proper Figure object with a Caption API and a way to capture that
    % in the conversion process - todo).
    \usepackage{caption}
    \DeclareCaptionLabelFormat{nolabel}{}
    \captionsetup{labelformat=nolabel}

    \usepackage{adjustbox} % Used to constrain images to a maximum size 
    \usepackage{xcolor} % Allow colors to be defined
    \usepackage{enumerate} % Needed for markdown enumerations to work
    \usepackage{geometry} % Used to adjust the document margins
    \usepackage{amsmath} % Equations
    \usepackage{amssymb} % Equations
    \usepackage{textcomp} % defines textquotesingle
    % Hack from http://tex.stackexchange.com/a/47451/13684:
    \AtBeginDocument{%
        \def\PYZsq{\textquotesingle}% Upright quotes in Pygmentized code
    }
    \usepackage{upquote} % Upright quotes for verbatim code
    \usepackage{eurosym} % defines \euro
    \usepackage[mathletters]{ucs} % Extended unicode (utf-8) support
    \usepackage[utf8x]{inputenc} % Allow utf-8 characters in the tex document
    \usepackage{fancyvrb} % verbatim replacement that allows latex
    \usepackage{grffile} % extends the file name processing of package graphics 
                         % to support a larger range 
    % The hyperref package gives us a pdf with properly built
    % internal navigation ('pdf bookmarks' for the table of contents,
    % internal cross-reference links, web links for URLs, etc.)
    \usepackage{hyperref}
    \usepackage{longtable} % longtable support required by pandoc >1.10
    \usepackage{booktabs}  % table support for pandoc > 1.12.2
    \usepackage[inline]{enumitem} % IRkernel/repr support (it uses the enumerate* environment)
    \usepackage[normalem]{ulem} % ulem is needed to support strikethroughs (\sout)
                                % normalem makes italics be italics, not underlines
    \usepackage{mathrsfs}
    

    
    
    % Colors for the hyperref package
    \definecolor{urlcolor}{rgb}{0,.145,.698}
    \definecolor{linkcolor}{rgb}{.71,0.21,0.01}
    \definecolor{citecolor}{rgb}{.12,.54,.11}

    % ANSI colors
    \definecolor{ansi-black}{HTML}{3E424D}
    \definecolor{ansi-black-intense}{HTML}{282C36}
    \definecolor{ansi-red}{HTML}{E75C58}
    \definecolor{ansi-red-intense}{HTML}{B22B31}
    \definecolor{ansi-green}{HTML}{00A250}
    \definecolor{ansi-green-intense}{HTML}{007427}
    \definecolor{ansi-yellow}{HTML}{DDB62B}
    \definecolor{ansi-yellow-intense}{HTML}{B27D12}
    \definecolor{ansi-blue}{HTML}{208FFB}
    \definecolor{ansi-blue-intense}{HTML}{0065CA}
    \definecolor{ansi-magenta}{HTML}{D160C4}
    \definecolor{ansi-magenta-intense}{HTML}{A03196}
    \definecolor{ansi-cyan}{HTML}{60C6C8}
    \definecolor{ansi-cyan-intense}{HTML}{258F8F}
    \definecolor{ansi-white}{HTML}{C5C1B4}
    \definecolor{ansi-white-intense}{HTML}{A1A6B2}
    \definecolor{ansi-default-inverse-fg}{HTML}{FFFFFF}
    \definecolor{ansi-default-inverse-bg}{HTML}{000000}

    % commands and environments needed by pandoc snippets
    % extracted from the output of `pandoc -s`
    \providecommand{\tightlist}{%
      \setlength{\itemsep}{0pt}\setlength{\parskip}{0pt}}
    \DefineVerbatimEnvironment{Highlighting}{Verbatim}{commandchars=\\\{\}}
    % Add ',fontsize=\small' for more characters per line
    \newenvironment{Shaded}{}{}
    \newcommand{\KeywordTok}[1]{\textcolor[rgb]{0.00,0.44,0.13}{\textbf{{#1}}}}
    \newcommand{\DataTypeTok}[1]{\textcolor[rgb]{0.56,0.13,0.00}{{#1}}}
    \newcommand{\DecValTok}[1]{\textcolor[rgb]{0.25,0.63,0.44}{{#1}}}
    \newcommand{\BaseNTok}[1]{\textcolor[rgb]{0.25,0.63,0.44}{{#1}}}
    \newcommand{\FloatTok}[1]{\textcolor[rgb]{0.25,0.63,0.44}{{#1}}}
    \newcommand{\CharTok}[1]{\textcolor[rgb]{0.25,0.44,0.63}{{#1}}}
    \newcommand{\StringTok}[1]{\textcolor[rgb]{0.25,0.44,0.63}{{#1}}}
    \newcommand{\CommentTok}[1]{\textcolor[rgb]{0.38,0.63,0.69}{\textit{{#1}}}}
    \newcommand{\OtherTok}[1]{\textcolor[rgb]{0.00,0.44,0.13}{{#1}}}
    \newcommand{\AlertTok}[1]{\textcolor[rgb]{1.00,0.00,0.00}{\textbf{{#1}}}}
    \newcommand{\FunctionTok}[1]{\textcolor[rgb]{0.02,0.16,0.49}{{#1}}}
    \newcommand{\RegionMarkerTok}[1]{{#1}}
    \newcommand{\ErrorTok}[1]{\textcolor[rgb]{1.00,0.00,0.00}{\textbf{{#1}}}}
    \newcommand{\NormalTok}[1]{{#1}}
    
    % Additional commands for more recent versions of Pandoc
    \newcommand{\ConstantTok}[1]{\textcolor[rgb]{0.53,0.00,0.00}{{#1}}}
    \newcommand{\SpecialCharTok}[1]{\textcolor[rgb]{0.25,0.44,0.63}{{#1}}}
    \newcommand{\VerbatimStringTok}[1]{\textcolor[rgb]{0.25,0.44,0.63}{{#1}}}
    \newcommand{\SpecialStringTok}[1]{\textcolor[rgb]{0.73,0.40,0.53}{{#1}}}
    \newcommand{\ImportTok}[1]{{#1}}
    \newcommand{\DocumentationTok}[1]{\textcolor[rgb]{0.73,0.13,0.13}{\textit{{#1}}}}
    \newcommand{\AnnotationTok}[1]{\textcolor[rgb]{0.38,0.63,0.69}{\textbf{\textit{{#1}}}}}
    \newcommand{\CommentVarTok}[1]{\textcolor[rgb]{0.38,0.63,0.69}{\textbf{\textit{{#1}}}}}
    \newcommand{\VariableTok}[1]{\textcolor[rgb]{0.10,0.09,0.49}{{#1}}}
    \newcommand{\ControlFlowTok}[1]{\textcolor[rgb]{0.00,0.44,0.13}{\textbf{{#1}}}}
    \newcommand{\OperatorTok}[1]{\textcolor[rgb]{0.40,0.40,0.40}{{#1}}}
    \newcommand{\BuiltInTok}[1]{{#1}}
    \newcommand{\ExtensionTok}[1]{{#1}}
    \newcommand{\PreprocessorTok}[1]{\textcolor[rgb]{0.74,0.48,0.00}{{#1}}}
    \newcommand{\AttributeTok}[1]{\textcolor[rgb]{0.49,0.56,0.16}{{#1}}}
    \newcommand{\InformationTok}[1]{\textcolor[rgb]{0.38,0.63,0.69}{\textbf{\textit{{#1}}}}}
    \newcommand{\WarningTok}[1]{\textcolor[rgb]{0.38,0.63,0.69}{\textbf{\textit{{#1}}}}}
    
    
    % Define a nice break command that doesn't care if a line doesn't already
    % exist.
    \def\br{\hspace*{\fill} \\* }
    % Math Jax compatibility definitions
    \def\gt{>}
    \def\lt{<}
    \let\Oldtex\TeX
    \let\Oldlatex\LaTeX
    \renewcommand{\TeX}{\textrm{\Oldtex}}
    \renewcommand{\LaTeX}{\textrm{\Oldlatex}}
    % Document parameters
    % Document title
    \title{ETHANWALKER-404-FinalProject}
    
    
    
    
    

    % Pygments definitions
    
\makeatletter
\def\PY@reset{\let\PY@it=\relax \let\PY@bf=\relax%
    \let\PY@ul=\relax \let\PY@tc=\relax%
    \let\PY@bc=\relax \let\PY@ff=\relax}
\def\PY@tok#1{\csname PY@tok@#1\endcsname}
\def\PY@toks#1+{\ifx\relax#1\empty\else%
    \PY@tok{#1}\expandafter\PY@toks\fi}
\def\PY@do#1{\PY@bc{\PY@tc{\PY@ul{%
    \PY@it{\PY@bf{\PY@ff{#1}}}}}}}
\def\PY#1#2{\PY@reset\PY@toks#1+\relax+\PY@do{#2}}

\expandafter\def\csname PY@tok@w\endcsname{\def\PY@tc##1{\textcolor[rgb]{0.73,0.73,0.73}{##1}}}
\expandafter\def\csname PY@tok@c\endcsname{\let\PY@it=\textit\def\PY@tc##1{\textcolor[rgb]{0.25,0.50,0.50}{##1}}}
\expandafter\def\csname PY@tok@cp\endcsname{\def\PY@tc##1{\textcolor[rgb]{0.74,0.48,0.00}{##1}}}
\expandafter\def\csname PY@tok@k\endcsname{\let\PY@bf=\textbf\def\PY@tc##1{\textcolor[rgb]{0.00,0.50,0.00}{##1}}}
\expandafter\def\csname PY@tok@kp\endcsname{\def\PY@tc##1{\textcolor[rgb]{0.00,0.50,0.00}{##1}}}
\expandafter\def\csname PY@tok@kt\endcsname{\def\PY@tc##1{\textcolor[rgb]{0.69,0.00,0.25}{##1}}}
\expandafter\def\csname PY@tok@o\endcsname{\def\PY@tc##1{\textcolor[rgb]{0.40,0.40,0.40}{##1}}}
\expandafter\def\csname PY@tok@ow\endcsname{\let\PY@bf=\textbf\def\PY@tc##1{\textcolor[rgb]{0.67,0.13,1.00}{##1}}}
\expandafter\def\csname PY@tok@nb\endcsname{\def\PY@tc##1{\textcolor[rgb]{0.00,0.50,0.00}{##1}}}
\expandafter\def\csname PY@tok@nf\endcsname{\def\PY@tc##1{\textcolor[rgb]{0.00,0.00,1.00}{##1}}}
\expandafter\def\csname PY@tok@nc\endcsname{\let\PY@bf=\textbf\def\PY@tc##1{\textcolor[rgb]{0.00,0.00,1.00}{##1}}}
\expandafter\def\csname PY@tok@nn\endcsname{\let\PY@bf=\textbf\def\PY@tc##1{\textcolor[rgb]{0.00,0.00,1.00}{##1}}}
\expandafter\def\csname PY@tok@ne\endcsname{\let\PY@bf=\textbf\def\PY@tc##1{\textcolor[rgb]{0.82,0.25,0.23}{##1}}}
\expandafter\def\csname PY@tok@nv\endcsname{\def\PY@tc##1{\textcolor[rgb]{0.10,0.09,0.49}{##1}}}
\expandafter\def\csname PY@tok@no\endcsname{\def\PY@tc##1{\textcolor[rgb]{0.53,0.00,0.00}{##1}}}
\expandafter\def\csname PY@tok@nl\endcsname{\def\PY@tc##1{\textcolor[rgb]{0.63,0.63,0.00}{##1}}}
\expandafter\def\csname PY@tok@ni\endcsname{\let\PY@bf=\textbf\def\PY@tc##1{\textcolor[rgb]{0.60,0.60,0.60}{##1}}}
\expandafter\def\csname PY@tok@na\endcsname{\def\PY@tc##1{\textcolor[rgb]{0.49,0.56,0.16}{##1}}}
\expandafter\def\csname PY@tok@nt\endcsname{\let\PY@bf=\textbf\def\PY@tc##1{\textcolor[rgb]{0.00,0.50,0.00}{##1}}}
\expandafter\def\csname PY@tok@nd\endcsname{\def\PY@tc##1{\textcolor[rgb]{0.67,0.13,1.00}{##1}}}
\expandafter\def\csname PY@tok@s\endcsname{\def\PY@tc##1{\textcolor[rgb]{0.73,0.13,0.13}{##1}}}
\expandafter\def\csname PY@tok@sd\endcsname{\let\PY@it=\textit\def\PY@tc##1{\textcolor[rgb]{0.73,0.13,0.13}{##1}}}
\expandafter\def\csname PY@tok@si\endcsname{\let\PY@bf=\textbf\def\PY@tc##1{\textcolor[rgb]{0.73,0.40,0.53}{##1}}}
\expandafter\def\csname PY@tok@se\endcsname{\let\PY@bf=\textbf\def\PY@tc##1{\textcolor[rgb]{0.73,0.40,0.13}{##1}}}
\expandafter\def\csname PY@tok@sr\endcsname{\def\PY@tc##1{\textcolor[rgb]{0.73,0.40,0.53}{##1}}}
\expandafter\def\csname PY@tok@ss\endcsname{\def\PY@tc##1{\textcolor[rgb]{0.10,0.09,0.49}{##1}}}
\expandafter\def\csname PY@tok@sx\endcsname{\def\PY@tc##1{\textcolor[rgb]{0.00,0.50,0.00}{##1}}}
\expandafter\def\csname PY@tok@m\endcsname{\def\PY@tc##1{\textcolor[rgb]{0.40,0.40,0.40}{##1}}}
\expandafter\def\csname PY@tok@gh\endcsname{\let\PY@bf=\textbf\def\PY@tc##1{\textcolor[rgb]{0.00,0.00,0.50}{##1}}}
\expandafter\def\csname PY@tok@gu\endcsname{\let\PY@bf=\textbf\def\PY@tc##1{\textcolor[rgb]{0.50,0.00,0.50}{##1}}}
\expandafter\def\csname PY@tok@gd\endcsname{\def\PY@tc##1{\textcolor[rgb]{0.63,0.00,0.00}{##1}}}
\expandafter\def\csname PY@tok@gi\endcsname{\def\PY@tc##1{\textcolor[rgb]{0.00,0.63,0.00}{##1}}}
\expandafter\def\csname PY@tok@gr\endcsname{\def\PY@tc##1{\textcolor[rgb]{1.00,0.00,0.00}{##1}}}
\expandafter\def\csname PY@tok@ge\endcsname{\let\PY@it=\textit}
\expandafter\def\csname PY@tok@gs\endcsname{\let\PY@bf=\textbf}
\expandafter\def\csname PY@tok@gp\endcsname{\let\PY@bf=\textbf\def\PY@tc##1{\textcolor[rgb]{0.00,0.00,0.50}{##1}}}
\expandafter\def\csname PY@tok@go\endcsname{\def\PY@tc##1{\textcolor[rgb]{0.53,0.53,0.53}{##1}}}
\expandafter\def\csname PY@tok@gt\endcsname{\def\PY@tc##1{\textcolor[rgb]{0.00,0.27,0.87}{##1}}}
\expandafter\def\csname PY@tok@err\endcsname{\def\PY@bc##1{\setlength{\fboxsep}{0pt}\fcolorbox[rgb]{1.00,0.00,0.00}{1,1,1}{\strut ##1}}}
\expandafter\def\csname PY@tok@kc\endcsname{\let\PY@bf=\textbf\def\PY@tc##1{\textcolor[rgb]{0.00,0.50,0.00}{##1}}}
\expandafter\def\csname PY@tok@kd\endcsname{\let\PY@bf=\textbf\def\PY@tc##1{\textcolor[rgb]{0.00,0.50,0.00}{##1}}}
\expandafter\def\csname PY@tok@kn\endcsname{\let\PY@bf=\textbf\def\PY@tc##1{\textcolor[rgb]{0.00,0.50,0.00}{##1}}}
\expandafter\def\csname PY@tok@kr\endcsname{\let\PY@bf=\textbf\def\PY@tc##1{\textcolor[rgb]{0.00,0.50,0.00}{##1}}}
\expandafter\def\csname PY@tok@bp\endcsname{\def\PY@tc##1{\textcolor[rgb]{0.00,0.50,0.00}{##1}}}
\expandafter\def\csname PY@tok@fm\endcsname{\def\PY@tc##1{\textcolor[rgb]{0.00,0.00,1.00}{##1}}}
\expandafter\def\csname PY@tok@vc\endcsname{\def\PY@tc##1{\textcolor[rgb]{0.10,0.09,0.49}{##1}}}
\expandafter\def\csname PY@tok@vg\endcsname{\def\PY@tc##1{\textcolor[rgb]{0.10,0.09,0.49}{##1}}}
\expandafter\def\csname PY@tok@vi\endcsname{\def\PY@tc##1{\textcolor[rgb]{0.10,0.09,0.49}{##1}}}
\expandafter\def\csname PY@tok@vm\endcsname{\def\PY@tc##1{\textcolor[rgb]{0.10,0.09,0.49}{##1}}}
\expandafter\def\csname PY@tok@sa\endcsname{\def\PY@tc##1{\textcolor[rgb]{0.73,0.13,0.13}{##1}}}
\expandafter\def\csname PY@tok@sb\endcsname{\def\PY@tc##1{\textcolor[rgb]{0.73,0.13,0.13}{##1}}}
\expandafter\def\csname PY@tok@sc\endcsname{\def\PY@tc##1{\textcolor[rgb]{0.73,0.13,0.13}{##1}}}
\expandafter\def\csname PY@tok@dl\endcsname{\def\PY@tc##1{\textcolor[rgb]{0.73,0.13,0.13}{##1}}}
\expandafter\def\csname PY@tok@s2\endcsname{\def\PY@tc##1{\textcolor[rgb]{0.73,0.13,0.13}{##1}}}
\expandafter\def\csname PY@tok@sh\endcsname{\def\PY@tc##1{\textcolor[rgb]{0.73,0.13,0.13}{##1}}}
\expandafter\def\csname PY@tok@s1\endcsname{\def\PY@tc##1{\textcolor[rgb]{0.73,0.13,0.13}{##1}}}
\expandafter\def\csname PY@tok@mb\endcsname{\def\PY@tc##1{\textcolor[rgb]{0.40,0.40,0.40}{##1}}}
\expandafter\def\csname PY@tok@mf\endcsname{\def\PY@tc##1{\textcolor[rgb]{0.40,0.40,0.40}{##1}}}
\expandafter\def\csname PY@tok@mh\endcsname{\def\PY@tc##1{\textcolor[rgb]{0.40,0.40,0.40}{##1}}}
\expandafter\def\csname PY@tok@mi\endcsname{\def\PY@tc##1{\textcolor[rgb]{0.40,0.40,0.40}{##1}}}
\expandafter\def\csname PY@tok@il\endcsname{\def\PY@tc##1{\textcolor[rgb]{0.40,0.40,0.40}{##1}}}
\expandafter\def\csname PY@tok@mo\endcsname{\def\PY@tc##1{\textcolor[rgb]{0.40,0.40,0.40}{##1}}}
\expandafter\def\csname PY@tok@ch\endcsname{\let\PY@it=\textit\def\PY@tc##1{\textcolor[rgb]{0.25,0.50,0.50}{##1}}}
\expandafter\def\csname PY@tok@cm\endcsname{\let\PY@it=\textit\def\PY@tc##1{\textcolor[rgb]{0.25,0.50,0.50}{##1}}}
\expandafter\def\csname PY@tok@cpf\endcsname{\let\PY@it=\textit\def\PY@tc##1{\textcolor[rgb]{0.25,0.50,0.50}{##1}}}
\expandafter\def\csname PY@tok@c1\endcsname{\let\PY@it=\textit\def\PY@tc##1{\textcolor[rgb]{0.25,0.50,0.50}{##1}}}
\expandafter\def\csname PY@tok@cs\endcsname{\let\PY@it=\textit\def\PY@tc##1{\textcolor[rgb]{0.25,0.50,0.50}{##1}}}

\def\PYZbs{\char`\\}
\def\PYZus{\char`\_}
\def\PYZob{\char`\{}
\def\PYZcb{\char`\}}
\def\PYZca{\char`\^}
\def\PYZam{\char`\&}
\def\PYZlt{\char`\<}
\def\PYZgt{\char`\>}
\def\PYZsh{\char`\#}
\def\PYZpc{\char`\%}
\def\PYZdl{\char`\$}
\def\PYZhy{\char`\-}
\def\PYZsq{\char`\'}
\def\PYZdq{\char`\"}
\def\PYZti{\char`\~}
% for compatibility with earlier versions
\def\PYZat{@}
\def\PYZlb{[}
\def\PYZrb{]}
\makeatother


    % Exact colors from NB
    \definecolor{incolor}{rgb}{0.0, 0.0, 0.5}
    \definecolor{outcolor}{rgb}{0.545, 0.0, 0.0}



    
    % Prevent overflowing lines due to hard-to-break entities
    \sloppy 
    % Setup hyperref package
    \hypersetup{
      breaklinks=true,  % so long urls are correctly broken across lines
      colorlinks=true,
      urlcolor=urlcolor,
      linkcolor=linkcolor,
      citecolor=citecolor,
      }
    % Slightly bigger margins than the latex defaults
    
    \geometry{verbose,tmargin=1in,bmargin=1in,lmargin=1in,rmargin=1in}
    
    

    \begin{document}
    
    
    \maketitle
    
    

    
    \begin{Verbatim}[commandchars=\\\{\}]
/home/ethan/anaconda3/lib/python3.7/site-packages/dask/config.py:168: YAMLLoadWarning: calling yaml.load() without Loader={\ldots} is deprecated, as the default Loader is unsafe. Please read https://msg.pyyaml.org/load for full details.
  data = yaml.load(f.read()) or \{\}
/home/ethan/anaconda3/lib/python3.7/site-packages/distributed/config.py:20: YAMLLoadWarning: calling yaml.load() without Loader={\ldots} is deprecated, as the default Loader is unsafe. Please read https://msg.pyyaml.org/load for full details.
  defaults = yaml.load(f)

    \end{Verbatim}

    \hypertarget{abstract}{%
\section{Abstract}\label{abstract}}

Machine Learning is becomeing a more prevelant tool in the world of
criminal justice. Often it is used to predict who will commit a crime or
where crimes may occur. Seldom is it used to regulate the criminal
justice system, however. In this project I examine prison inmate data
and determine what machine learning techniques are effective at
detecting the racial bias that has been shown to exist in this data. In
this report I find ---------

\hypertarget{problem-statement-and-motivation}{%
\section{Problem Statement and
Motivation}\label{problem-statement-and-motivation}}

Last semester I took a look a data set containing the information of
more than 7.5 million individuals that have been processed by the
criminal justice system. I found that racial minorities were more likely
to receive extreme sentences, agreeing with existing research around
bias in the criminal justice system. In this project I will be exploring
the data from a machine learning perspective. My goal is to determine if
this data can be classified in such a way that is predictive of race.
The idea is that perhaps racial bias can be detected in various systems
by seeing how effective different machine learning techniques are at
classifying an inmate's by race given their data.

This is an unconventional way to approach criminal justice data with
machine learning. Often we see machine learning being used to attempt to
determine who might be a criminal or where crimal activity may occur
using social media data and other public information, which may include
data the government owns, but which is not available to te public. These
approaches often ignore or discount the ways that these techniques may
disproportionately affect people of color and the poor. Many
organizations have made official statements reguarding the use of
machine learning in this way, often called predicitive policing. The
ACLU for example released a statement listing civil rights related
concerns about predicitive policing which was signed by several civil
rights origanizations includeing the NAACP
\href{https://www.aclu.org/other/statement-concern-about-predictive-policing-aclu-and-16-civil-rights-privacy-racial-justice}{{[}1{]}}.

My objective is to go against the predictive policing paradigm and use
machine learning to benefit these negatively affected classes of people
by using machine learning as a diagnostic tool. If it can be shown that
certain machine learning techniques are effective at classifying inmates
by race given incarceration related information, then we can inform
policies that will attempt to correct for these systemic racial biases.

    \hypertarget{data}{%
\section{Data}\label{data}}

\hypertarget{source-and-credibility}{%
\subsection{Source and Credibility}\label{source-and-credibility}}

The data that I will be using in this analysis is from one source. It is
a
\href{https://catalog.data.gov/dataset/sentenced-inmates-in-correctional-facilities}{database}
hosted on \href{https://www.data.gov}{Data.gov} and maintained by the
State of Connecticut Department of Corrections. This source is highly
credible because it is a primary source for the data. This organization
is an official government agency which collects, maintains, and reports
on this data.

\hypertarget{gathering-and-cleaning}{%
\subsection{Gathering and Cleaning}\label{gathering-and-cleaning}}

All the data which I am using in this report are freely available to the
public. Collection and cleaning was relatively simple as the source data
was well maintained. The file that I obtained from the Connecticut
Department of Corrections is a very well maintained database. The
largest issue I had with this file was mild inconsistency with the way
in which certain data was encoded (ex. race was encoded as both
\texttt{WHITE} and \texttt{WHITE\textbackslash{}t}). The file is

\begin{verbatim}
individuals.csv.
\end{verbatim}

\hypertarget{about-the-data}{%
\subsection{About the Data}\label{about-the-data}}

This data set contains individual information for 7.77 million people
that have been processed by the justice system and recorded by the
Connecticut Department of Corrections. Each individual is recorded along
with their age, gender, race, offense, and sentence length, among other
things.

Because there is so much to consider in what is found in the data set, I
chose not to feature engineer as to avoid unneeded complexity.

The sample sizes among different races that are found in the Connecticut
Department of Justice data are not similar. The sample size for American
Indians and Asians is much smaller than that of Whites, Hispanics, and
Blacks, hence we may see some irregular outcomes in the analysis related
to these racial groups.

    \begin{Verbatim}[commandchars=\\\{\}]
Sample size for Blacks: 3287596
Sample size for Whites: 2393949
Sample size for Hispanic: 2039297
Sample size for American Indian: 21133
Sample size for Asian: 35660

    \end{Verbatim}

    \hypertarget{possible-questions}{%
\section{Possible questions}\label{possible-questions}}

can we use ml techniques to correctly classify this data? which ones
fail and why? can we create a predictive model for sentence lengths?
should sentencing be offloaded to a ml algorithm? what does it mean to
have an effective classfifier for this data set.

    \hypertarget{methods}{%
\section{Methods}\label{methods}}

Before I begin discussing the methods that I did use, I will talk about
some of the methods that I did not use. There are manay techniques that
are not applicable to this data. For example, since this data set is not
a time series models like ARMA and HMM are not applicable here.

A method that I attempted to use, but found little success with were
dimension reduction tools like PCA, T-SNE, and UMAP. This dimension
reduction would have been helpful, especially because once one-hot
encoded this data becomes extremely high dimensional. However PCA,
T-SNE, and UMAP all failed to create meaningful clusters, so I abandoned
the the pursuit of clustering early on. Perhaps some kernel methods
would have been helpful in this endeavor, however I could not find a
kernel that could create a metric on crimes and I dp not feel qualified
to write one myself.

Below I have images of my attempt to use PCA, T-SNE, and UMAP to cluster
the data. It is apperent that it simply is not effective.

    \begin{center}
    \adjustimage{max size={0.9\linewidth}{0.9\paperheight}}{ETHANWALKER-404-FinalProject_files/ETHANWALKER-404-FinalProject_8_0.png}
    \end{center}
    { \hspace*{\fill} \\}
    
    \begin{Verbatim}[commandchars=\\\{\}]
/home/ethan/anaconda3/lib/python3.7/site-packages/sklearn/manifold/spectral\_embedding\_.py:237: UserWarning: Graph is not fully connected, spectral embedding may not work as expected.
  warnings.warn("Graph is not fully connected, spectral embedding"

    \end{Verbatim}

    \begin{center}
    \adjustimage{max size={0.9\linewidth}{0.9\paperheight}}{ETHANWALKER-404-FinalProject_files/ETHANWALKER-404-FinalProject_9_1.png}
    \end{center}
    { \hspace*{\fill} \\}
    
    \begin{Verbatim}[commandchars=\\\{\}]
/home/ethan/anaconda3/lib/python3.7/site-packages/sklearn/manifold/spectral\_embedding\_.py:237: UserWarning: Graph is not fully connected, spectral embedding may not work as expected.
  warnings.warn("Graph is not fully connected, spectral embedding"

    \end{Verbatim}

    \begin{center}
    \adjustimage{max size={0.9\linewidth}{0.9\paperheight}}{ETHANWALKER-404-FinalProject_files/ETHANWALKER-404-FinalProject_10_1.png}
    \end{center}
    { \hspace*{\fill} \\}
    
    \begin{Verbatim}[commandchars=\\\{\}]
/home/ethan/anaconda3/lib/python3.7/site-packages/sklearn/manifold/spectral\_embedding\_.py:237: UserWarning: Graph is not fully connected, spectral embedding may not work as expected.
  warnings.warn("Graph is not fully connected, spectral embedding"

    \end{Verbatim}

    \begin{center}
    \adjustimage{max size={0.9\linewidth}{0.9\paperheight}}{ETHANWALKER-404-FinalProject_files/ETHANWALKER-404-FinalProject_11_1.png}
    \end{center}
    { \hspace*{\fill} \\}
    
    \begin{center}
    \adjustimage{max size={0.9\linewidth}{0.9\paperheight}}{ETHANWALKER-404-FinalProject_files/ETHANWALKER-404-FinalProject_12_0.png}
    \end{center}
    { \hspace*{\fill} \\}
    
    This is only a small sample of things that I tried to do to get useful
clusterings given the dimension reduction, though it is representitive
of the results I found

    \hypertarget{ensamble-methods}{%
\subsection{Ensamble Methods}\label{ensamble-methods}}

Ensamble methods seemed imediately like the methods that I should be
using in this project. I did not intially expect to successfully
classify this data, however as I learned about how ensamble classifiers
worked I became more confident that I would get decent results.

The primary reason that I though I would not get good results has to do
with how I identified racial bias in the data. In my previous project I
determined that there was racial bias in the criminal justice system
based on

\hypertarget{random-forest-classifier}{%
\subsubsection{Random Forest
Classifier}\label{random-forest-classifier}}

    \begin{Verbatim}[commandchars=\\\{\}]
{\color{incolor}In [{\color{incolor}44}]:} 
\end{Verbatim}

    \begin{Verbatim}[commandchars=\\\{\}]
{\color{incolor}In [{\color{incolor}46}]:} \PY{n}{samp} \PY{o}{=} \PY{n}{df}\PY{o}{.}\PY{n}{sample}\PY{p}{(}\PY{l+m+mi}{20000}\PY{p}{)}
         \PY{n}{samp}\PY{o}{.}\PY{n}{RACE} \PY{o}{=} \PY{n}{pd}\PY{o}{.}\PY{n}{factorize}\PY{p}{(}\PY{n}{samp}\PY{p}{[}\PY{l+s+s1}{\PYZsq{}}\PY{l+s+s1}{RACE}\PY{l+s+s1}{\PYZsq{}}\PY{p}{]}\PY{p}{)}\PY{p}{[}\PY{l+m+mi}{0}\PY{p}{]} \PY{o}{+} \PY{l+m+mi}{1}
         \PY{n}{samp}\PY{o}{.}\PY{n}{GENDER} \PY{o}{=} \PY{n}{pd}\PY{o}{.}\PY{n}{factorize}\PY{p}{(}\PY{n}{samp}\PY{p}{[}\PY{l+s+s1}{\PYZsq{}}\PY{l+s+s1}{GENDER}\PY{l+s+s1}{\PYZsq{}}\PY{p}{]}\PY{p}{)}\PY{p}{[}\PY{l+m+mi}{0}\PY{p}{]} \PY{o}{+} \PY{l+m+mi}{1}
         \PY{n}{samp}\PY{o}{.}\PY{n}{OFFENSE} \PY{o}{=} \PY{n}{pd}\PY{o}{.}\PY{n}{factorize}\PY{p}{(}\PY{n}{samp}\PY{p}{[}\PY{l+s+s1}{\PYZsq{}}\PY{l+s+s1}{OFFENSE}\PY{l+s+s1}{\PYZsq{}}\PY{p}{]}\PY{p}{)}\PY{p}{[}\PY{l+m+mi}{0}\PY{p}{]} \PY{o}{+} \PY{l+m+mi}{1}
         \PY{n}{samp}\PY{o}{.}\PY{n}{DETAINER} \PY{o}{=} \PY{n}{pd}\PY{o}{.}\PY{n}{factorize}\PY{p}{(}\PY{n}{samp}\PY{p}{[}\PY{l+s+s1}{\PYZsq{}}\PY{l+s+s1}{DETAINER}\PY{l+s+s1}{\PYZsq{}}\PY{p}{]}\PY{p}{)}\PY{p}{[}\PY{l+m+mi}{0}\PY{p}{]} \PY{o}{+} \PY{l+m+mi}{1}
         \PY{n}{samp}\PY{o}{.}\PY{n}{FACILITY} \PY{o}{=} \PY{n}{pd}\PY{o}{.}\PY{n}{factorize}\PY{p}{(}\PY{n}{samp}\PY{p}{[}\PY{l+s+s1}{\PYZsq{}}\PY{l+s+s1}{FACILITY}\PY{l+s+s1}{\PYZsq{}}\PY{p}{]}\PY{p}{)}\PY{p}{[}\PY{l+m+mi}{0}\PY{p}{]} \PY{o}{+} \PY{l+m+mi}{1}
         \PY{n}{samp\PYZus{}y} \PY{o}{=} \PY{n}{samp}\PY{o}{.}\PY{n}{RACE}
         \PY{n}{samp\PYZus{}X} \PY{o}{=} \PY{n}{samp}\PY{p}{[}\PY{p}{[}\PY{l+s+s1}{\PYZsq{}}\PY{l+s+s1}{GENDER}\PY{l+s+s1}{\PYZsq{}}\PY{p}{,}\PY{l+s+s1}{\PYZsq{}}\PY{l+s+s1}{AGE}\PY{l+s+s1}{\PYZsq{}}\PY{p}{,}\PY{l+s+s1}{\PYZsq{}}\PY{l+s+s1}{OFFENSE}\PY{l+s+s1}{\PYZsq{}}\PY{p}{,}\PY{l+s+s1}{\PYZsq{}}\PY{l+s+s1}{FACILITY}\PY{l+s+s1}{\PYZsq{}}\PY{p}{,}\PY{l+s+s1}{\PYZsq{}}\PY{l+s+s1}{DETAINER}\PY{l+s+s1}{\PYZsq{}}\PY{p}{,}\PY{l+s+s1}{\PYZsq{}}\PY{l+s+s1}{SENTENCE DAYS}\PY{l+s+s1}{\PYZsq{}}\PY{p}{]}\PY{p}{]}
         
         \PY{n}{param\PYZus{}grid} \PY{o}{=} \PY{p}{\PYZob{}}
             \PY{l+s+s1}{\PYZsq{}}\PY{l+s+s1}{n\PYZus{}estimators}\PY{l+s+s1}{\PYZsq{}}\PY{p}{:} \PY{n}{np}\PY{o}{.}\PY{n}{arange}\PY{p}{(}\PY{l+m+mi}{100}\PY{p}{,}\PY{l+m+mi}{400}\PY{p}{,}\PY{l+m+mi}{20}\PY{p}{)}\PY{p}{,}
             \PY{l+s+s1}{\PYZsq{}}\PY{l+s+s1}{max\PYZus{}depth}\PY{l+s+s1}{\PYZsq{}}\PY{p}{:} \PY{n}{np}\PY{o}{.}\PY{n}{arange}\PY{p}{(}\PY{l+m+mi}{10}\PY{p}{,}\PY{l+m+mi}{100}\PY{p}{,}\PY{l+m+mi}{10}\PY{p}{)}\PY{p}{,}
             \PY{l+s+s1}{\PYZsq{}}\PY{l+s+s1}{max\PYZus{}features}\PY{l+s+s1}{\PYZsq{}}\PY{p}{:} \PY{n}{np}\PY{o}{.}\PY{n}{arange}\PY{p}{(}\PY{l+m+mi}{1}\PY{p}{,}\PY{l+m+mi}{6}\PY{p}{)}
         \PY{p}{\PYZcb{}}
         
         
         
         \PY{n}{clf} \PY{o}{=} \PY{n}{RandomForestClassifier}\PY{p}{(}\PY{n}{oob\PYZus{}score}\PY{o}{=}\PY{k+kc}{True}\PY{p}{)}
         
         \PY{n}{s} \PY{o}{=} \PY{n}{time}\PY{o}{.}\PY{n}{time}\PY{p}{(}\PY{p}{)}
         \PY{n}{clf} \PY{o}{=} \PY{n}{GridSearchCV}\PY{p}{(}\PY{n}{clf}\PY{p}{,} \PY{n}{param\PYZus{}grid}\PY{p}{,} \PY{n}{scoring}\PY{o}{=}\PY{k+kc}{None}\PY{p}{,} \PY{n}{cv}\PY{o}{=}\PY{l+m+mi}{5}\PY{p}{)}
         \PY{n}{clf} \PY{o}{=} \PY{n}{clf}\PY{o}{.}\PY{n}{fit}\PY{p}{(}\PY{n}{samp\PYZus{}X}\PY{p}{,} \PY{n}{samp\PYZus{}y}\PY{p}{)}
         \PY{n}{e} \PY{o}{=} \PY{n}{time}\PY{o}{.}\PY{n}{time}\PY{p}{(}\PY{p}{)}
         \PY{n+nb}{print}\PY{p}{(}\PY{n}{f}\PY{l+s+s1}{\PYZsq{}}\PY{l+s+s1}{time to train is }\PY{l+s+s1}{\PYZob{}}\PY{l+s+s1}{(e\PYZhy{}s)/60\PYZcb{} minutes}\PY{l+s+s1}{\PYZsq{}}\PY{p}{)}
         
         \PY{n}{clf} \PY{o}{=} \PY{n}{clf}\PY{o}{.}\PY{n}{best\PYZus{}estimator\PYZus{}}
         \PY{n}{clf} \PY{o}{=} \PY{n}{clf}\PY{o}{.}\PY{n}{fit}\PY{p}{(}\PY{n}{samp\PYZus{}X}\PY{p}{,} \PY{n}{samp\PYZus{}y}\PY{p}{)}
         \PY{n+nb}{print}\PY{p}{(}\PY{n}{f}\PY{l+s+s1}{\PYZsq{}}\PY{l+s+s1}{oob score is }\PY{l+s+si}{\PYZob{}clf.oob\PYZus{}score\PYZus{}\PYZcb{}}\PY{l+s+s1}{\PYZsq{}}\PY{p}{)}
         
         \PY{k}{with} \PY{n+nb}{open}\PY{p}{(}\PY{l+s+s1}{\PYZsq{}}\PY{l+s+s1}{RandomForestClf.pickle}\PY{l+s+s1}{\PYZsq{}}\PY{p}{,} \PY{l+s+s2}{\PYZdq{}}\PY{l+s+s2}{wb+}\PY{l+s+s2}{\PYZdq{}}\PY{p}{)} \PY{k}{as} \PY{n}{f}\PY{p}{:}
             \PY{n}{pickle}\PY{o}{.}\PY{n}{dump}\PY{p}{(}\PY{n}{clf}\PY{p}{,} \PY{n}{f}\PY{p}{)}
\end{Verbatim}

    \begin{Verbatim}[commandchars=\\\{\}]
time to train is 223.73655876318614 minutes
oob score is 0.70645

    \end{Verbatim}

    \begin{Verbatim}[commandchars=\\\{\}]
{\color{incolor}In [{\color{incolor}4}]:} \PY{k}{with} \PY{n+nb}{open}\PY{p}{(}\PY{l+s+s1}{\PYZsq{}}\PY{l+s+s1}{RandomForestClf.pickle}\PY{l+s+s1}{\PYZsq{}}\PY{p}{,}\PY{l+s+s2}{\PYZdq{}}\PY{l+s+s2}{rb}\PY{l+s+s2}{\PYZdq{}}\PY{p}{)} \PY{k}{as} \PY{n}{f}\PY{p}{:}
            \PY{n}{clf} \PY{o}{=} \PY{n}{pickle}\PY{o}{.}\PY{n}{load}\PY{p}{(}\PY{n}{f}\PY{p}{)}
        \PY{n+nb}{print}\PY{p}{(}\PY{n}{clf}\PY{o}{.}\PY{n}{oob\PYZus{}score\PYZus{}}\PY{p}{)}
        \PY{n+nb}{print}\PY{p}{(}\PY{n}{clf}\PY{o}{.}\PY{n}{feature\PYZus{}importances\PYZus{}}\PY{p}{)}
\end{Verbatim}

    \begin{Verbatim}[commandchars=\\\{\}]
/home/ethan/.local/lib/python3.7/site-packages/sklearn/base.py:306: UserWarning: Trying to unpickle estimator DecisionTreeClassifier from version 0.20.3 when using version 0.21.3. This might lead to breaking code or invalid results. Use at your own risk.
  UserWarning)

    \end{Verbatim}

    \begin{Verbatim}[commandchars=\\\{\}]
0.70645
[0.01022266 0.26942225 0.2124008  0.17300606 0.0338177  0.30113052]

    \end{Verbatim}

    \begin{Verbatim}[commandchars=\\\{\}]
/home/ethan/.local/lib/python3.7/site-packages/sklearn/base.py:306: UserWarning: Trying to unpickle estimator RandomForestClassifier from version 0.20.3 when using version 0.21.3. This might lead to breaking code or invalid results. Use at your own risk.
  UserWarning)

    \end{Verbatim}

    here we can see that the most important features are sentence length,
age, and offense. And the high OoB score is promising.

\begin{Verbatim}[commandchars=\\\{\}]
{\color{outcolor}Out[{\color{outcolor}5}]:} 0.420315
\end{Verbatim}
            
    However the score is not great. Random chance would be .2, so it does do
better than chance, though not much better.

    \hypertarget{gradient-descent-boosted-classification}{%
\subsection{Gradient Descent Boosted
Classification}\label{gradient-descent-boosted-classification}}

    \begin{Verbatim}[commandchars=\\\{\}]
{\color{incolor}In [{\color{incolor}35}]:} 
\end{Verbatim}

\begin{Verbatim}[commandchars=\\\{\}]
{\color{outcolor}Out[{\color{outcolor}6}]:} 0.296365
\end{Verbatim}
            
    \begin{Verbatim}[commandchars=\\\{\}]
[0.02090027 0.24716962 0.25775903 0.10433119 0.03748648 0.33235341]

    \end{Verbatim}

    \begin{Verbatim}[commandchars=\\\{\}]
/home/ethan/.local/lib/python3.7/site-packages/sklearn/base.py:306: UserWarning: Trying to unpickle estimator DecisionTreeRegressor from version 0.20.3 when using version 0.21.3. This might lead to breaking code or invalid results. Use at your own risk.
  UserWarning)
/home/ethan/.local/lib/python3.7/site-packages/sklearn/base.py:306: UserWarning: Trying to unpickle estimator GradientBoostingClassifier from version 0.20.3 when using version 0.21.3. This might lead to breaking code or invalid results. Use at your own risk.
  UserWarning)

    \end{Verbatim}

    Here we see the same feature importances as we did with the random
forest

\begin{Verbatim}[commandchars=\\\{\}]
{\color{outcolor}Out[{\color{outcolor}39}]:} \{'criterion': 'friedman\_mse',
          'init': None,
          'learning\_rate': 0.34,
          'loss': 'deviance',
          'max\_depth': 4,
          'max\_features': None,
          'max\_leaf\_nodes': None,
          'min\_impurity\_decrease': 0.0,
          'min\_impurity\_split': None,
          'min\_samples\_leaf': 1,
          'min\_samples\_split': 2,
          'min\_weight\_fraction\_leaf': 0.0,
          'n\_estimators': 600,
          'n\_iter\_no\_change': None,
          'presort': 'auto',
          'random\_state': None,
          'subsample': 1.0,
          'tol': 0.0001,
          'validation\_fraction': 0.1,
          'verbose': 0,
          'warm\_start': False\}
\end{Verbatim}
            
    \hypertarget{xg-boost}{%
\subsection{XG Boost}\label{xg-boost}}

    \begin{Verbatim}[commandchars=\\\{\}]
time was 1.7242353409528732 hours

    \end{Verbatim}

    \begin{Verbatim}[commandchars=\\\{\}]
0.41374
[0.2596316  0.132617   0.1624777  0.12035192 0.18768501 0.13723671]

    \end{Verbatim}

    This score is very promiseing since the score is about double chance.
There actually is a lot of correct classification going on here.

Interestingly however, it seems that the feature importances are very
different for this model. Gender is the most important feature nad every
other feature is about equally important.

    \hypertarget{k-nearest-neighbors}{%
\subsection{K-Nearest Neighbors}\label{k-nearest-neighbors}}

    \begin{Verbatim}[commandchars=\\\{\}]
{\color{incolor}In [{\color{incolor} }]:} 
\end{Verbatim}

    \hypertarget{kd-trees}{%
\subsection{KD Trees}\label{kd-trees}}

    \begin{Verbatim}[commandchars=\\\{\}]
{\color{incolor}In [{\color{incolor} }]:} 
\end{Verbatim}

    \hypertarget{gda}{%
\section{GDA}\label{gda}}

    \begin{Verbatim}[commandchars=\\\{\}]
{\color{incolor}In [{\color{incolor} }]:} 
\end{Verbatim}

    \hypertarget{results}{%
\section{Results}\label{results}}

\hypertarget{analysis}{%
\section{Analysis}\label{analysis}}

\hypertarget{conclusion}{%
\section{Conclusion}\label{conclusion}}

    \hypertarget{references}{%
\section{References}\label{references}}

\href{https://www.aclu.org/other/statement-concern-about-predictive-policing-aclu-and-16-civil-rights-privacy-racial-justice}{{[}1{]}}
Statement Of Concern About Predictive Policing By Aclu and 16 Civil
Rights Privacy, Racial Justice, and Technology Organizations
https://www.aclu.org/other/statement-concern-about-predictive-policing-aclu-and-16-civil-rights-privacy-racial-justice


    % Add a bibliography block to the postdoc
    
    
    
    \end{document}
