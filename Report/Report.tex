
% Default to the notebook output style

    


% Inherit from the specified cell style.




    
\documentclass[11pt]{article}

    
    
    \usepackage[T1]{fontenc}
    % Nicer default font (+ math font) than Computer Modern for most use cases
    \usepackage{mathpazo}

    % Basic figure setup, for now with no caption control since it's done
    % automatically by Pandoc (which extracts ![](path) syntax from Markdown).
    \usepackage{graphicx}
    % We will generate all images so they have a width \maxwidth. This means
    % that they will get their normal width if they fit onto the page, but
    % are scaled down if they would overflow the margins.
    \makeatletter
    \def\maxwidth{\ifdim\Gin@nat@width>\linewidth\linewidth
    \else\Gin@nat@width\fi}
    \makeatother
    \let\Oldincludegraphics\includegraphics
    % Set max figure width to be 80% of text width, for now hardcoded.
    \renewcommand{\includegraphics}[1]{\Oldincludegraphics[width=.8\maxwidth]{#1}}
    % Ensure that by default, figures have no caption (until we provide a
    % proper Figure object with a Caption API and a way to capture that
    % in the conversion process - todo).
    \usepackage{caption}
    \DeclareCaptionLabelFormat{nolabel}{}
    \captionsetup{labelformat=nolabel}

    \usepackage{adjustbox} % Used to constrain images to a maximum size 
    \usepackage{xcolor} % Allow colors to be defined
    \usepackage{enumerate} % Needed for markdown enumerations to work
    \usepackage{geometry} % Used to adjust the document margins
    \usepackage{amsmath} % Equations
    \usepackage{amssymb} % Equations
    \usepackage{textcomp} % defines textquotesingle
    % Hack from http://tex.stackexchange.com/a/47451/13684:
    \AtBeginDocument{%
        \def\PYZsq{\textquotesingle}% Upright quotes in Pygmentized code
    }
    \usepackage{upquote} % Upright quotes for verbatim code
    \usepackage{eurosym} % defines \euro
    \usepackage[mathletters]{ucs} % Extended unicode (utf-8) support
    \usepackage[utf8x]{inputenc} % Allow utf-8 characters in the tex document
    \usepackage{fancyvrb} % verbatim replacement that allows latex
    \usepackage{grffile} % extends the file name processing of package graphics 
                         % to support a larger range 
    % The hyperref package gives us a pdf with properly built
    % internal navigation ('pdf bookmarks' for the table of contents,
    % internal cross-reference links, web links for URLs, etc.)
    \usepackage{hyperref}
    \usepackage{longtable} % longtable support required by pandoc >1.10
    \usepackage{booktabs}  % table support for pandoc > 1.12.2
    \usepackage[inline]{enumitem} % IRkernel/repr support (it uses the enumerate* environment)
    \usepackage[normalem]{ulem} % ulem is needed to support strikethroughs (\sout)
                                % normalem makes italics be italics, not underlines
    \usepackage{mathrsfs}
    

    
    
    % Colors for the hyperref package
    \definecolor{urlcolor}{rgb}{0,.145,.698}
    \definecolor{linkcolor}{rgb}{.71,0.21,0.01}
    \definecolor{citecolor}{rgb}{.12,.54,.11}

    % ANSI colors
    \definecolor{ansi-black}{HTML}{3E424D}
    \definecolor{ansi-black-intense}{HTML}{282C36}
    \definecolor{ansi-red}{HTML}{E75C58}
    \definecolor{ansi-red-intense}{HTML}{B22B31}
    \definecolor{ansi-green}{HTML}{00A250}
    \definecolor{ansi-green-intense}{HTML}{007427}
    \definecolor{ansi-yellow}{HTML}{DDB62B}
    \definecolor{ansi-yellow-intense}{HTML}{B27D12}
    \definecolor{ansi-blue}{HTML}{208FFB}
    \definecolor{ansi-blue-intense}{HTML}{0065CA}
    \definecolor{ansi-magenta}{HTML}{D160C4}
    \definecolor{ansi-magenta-intense}{HTML}{A03196}
    \definecolor{ansi-cyan}{HTML}{60C6C8}
    \definecolor{ansi-cyan-intense}{HTML}{258F8F}
    \definecolor{ansi-white}{HTML}{C5C1B4}
    \definecolor{ansi-white-intense}{HTML}{A1A6B2}
    \definecolor{ansi-default-inverse-fg}{HTML}{FFFFFF}
    \definecolor{ansi-default-inverse-bg}{HTML}{000000}

    % commands and environments needed by pandoc snippets
    % extracted from the output of `pandoc -s`
    \providecommand{\tightlist}{%
      \setlength{\itemsep}{0pt}\setlength{\parskip}{0pt}}
    \DefineVerbatimEnvironment{Highlighting}{Verbatim}{commandchars=\\\{\}}
    % Add ',fontsize=\small' for more characters per line
    \newenvironment{Shaded}{}{}
    \newcommand{\KeywordTok}[1]{\textcolor[rgb]{0.00,0.44,0.13}{\textbf{{#1}}}}
    \newcommand{\DataTypeTok}[1]{\textcolor[rgb]{0.56,0.13,0.00}{{#1}}}
    \newcommand{\DecValTok}[1]{\textcolor[rgb]{0.25,0.63,0.44}{{#1}}}
    \newcommand{\BaseNTok}[1]{\textcolor[rgb]{0.25,0.63,0.44}{{#1}}}
    \newcommand{\FloatTok}[1]{\textcolor[rgb]{0.25,0.63,0.44}{{#1}}}
    \newcommand{\CharTok}[1]{\textcolor[rgb]{0.25,0.44,0.63}{{#1}}}
    \newcommand{\StringTok}[1]{\textcolor[rgb]{0.25,0.44,0.63}{{#1}}}
    \newcommand{\CommentTok}[1]{\textcolor[rgb]{0.38,0.63,0.69}{\textit{{#1}}}}
    \newcommand{\OtherTok}[1]{\textcolor[rgb]{0.00,0.44,0.13}{{#1}}}
    \newcommand{\AlertTok}[1]{\textcolor[rgb]{1.00,0.00,0.00}{\textbf{{#1}}}}
    \newcommand{\FunctionTok}[1]{\textcolor[rgb]{0.02,0.16,0.49}{{#1}}}
    \newcommand{\RegionMarkerTok}[1]{{#1}}
    \newcommand{\ErrorTok}[1]{\textcolor[rgb]{1.00,0.00,0.00}{\textbf{{#1}}}}
    \newcommand{\NormalTok}[1]{{#1}}
    
    % Additional commands for more recent versions of Pandoc
    \newcommand{\ConstantTok}[1]{\textcolor[rgb]{0.53,0.00,0.00}{{#1}}}
    \newcommand{\SpecialCharTok}[1]{\textcolor[rgb]{0.25,0.44,0.63}{{#1}}}
    \newcommand{\VerbatimStringTok}[1]{\textcolor[rgb]{0.25,0.44,0.63}{{#1}}}
    \newcommand{\SpecialStringTok}[1]{\textcolor[rgb]{0.73,0.40,0.53}{{#1}}}
    \newcommand{\ImportTok}[1]{{#1}}
    \newcommand{\DocumentationTok}[1]{\textcolor[rgb]{0.73,0.13,0.13}{\textit{{#1}}}}
    \newcommand{\AnnotationTok}[1]{\textcolor[rgb]{0.38,0.63,0.69}{\textbf{\textit{{#1}}}}}
    \newcommand{\CommentVarTok}[1]{\textcolor[rgb]{0.38,0.63,0.69}{\textbf{\textit{{#1}}}}}
    \newcommand{\VariableTok}[1]{\textcolor[rgb]{0.10,0.09,0.49}{{#1}}}
    \newcommand{\ControlFlowTok}[1]{\textcolor[rgb]{0.00,0.44,0.13}{\textbf{{#1}}}}
    \newcommand{\OperatorTok}[1]{\textcolor[rgb]{0.40,0.40,0.40}{{#1}}}
    \newcommand{\BuiltInTok}[1]{{#1}}
    \newcommand{\ExtensionTok}[1]{{#1}}
    \newcommand{\PreprocessorTok}[1]{\textcolor[rgb]{0.74,0.48,0.00}{{#1}}}
    \newcommand{\AttributeTok}[1]{\textcolor[rgb]{0.49,0.56,0.16}{{#1}}}
    \newcommand{\InformationTok}[1]{\textcolor[rgb]{0.38,0.63,0.69}{\textbf{\textit{{#1}}}}}
    \newcommand{\WarningTok}[1]{\textcolor[rgb]{0.38,0.63,0.69}{\textbf{\textit{{#1}}}}}
    
    
    % Define a nice break command that doesn't care if a line doesn't already
    % exist.
    \def\br{\hspace*{\fill} \\* }
    % Math Jax compatibility definitions
    \def\gt{>}
    \def\lt{<}
    \let\Oldtex\TeX
    \let\Oldlatex\LaTeX
    \renewcommand{\TeX}{\textrm{\Oldtex}}
    \renewcommand{\LaTeX}{\textrm{\Oldlatex}}
    % Document parameters
    % Document title
    \title{Report}
    
    
    
    
    

    % Pygments definitions
    
\makeatletter
\def\PY@reset{\let\PY@it=\relax \let\PY@bf=\relax%
    \let\PY@ul=\relax \let\PY@tc=\relax%
    \let\PY@bc=\relax \let\PY@ff=\relax}
\def\PY@tok#1{\csname PY@tok@#1\endcsname}
\def\PY@toks#1+{\ifx\relax#1\empty\else%
    \PY@tok{#1}\expandafter\PY@toks\fi}
\def\PY@do#1{\PY@bc{\PY@tc{\PY@ul{%
    \PY@it{\PY@bf{\PY@ff{#1}}}}}}}
\def\PY#1#2{\PY@reset\PY@toks#1+\relax+\PY@do{#2}}

\expandafter\def\csname PY@tok@w\endcsname{\def\PY@tc##1{\textcolor[rgb]{0.73,0.73,0.73}{##1}}}
\expandafter\def\csname PY@tok@c\endcsname{\let\PY@it=\textit\def\PY@tc##1{\textcolor[rgb]{0.25,0.50,0.50}{##1}}}
\expandafter\def\csname PY@tok@cp\endcsname{\def\PY@tc##1{\textcolor[rgb]{0.74,0.48,0.00}{##1}}}
\expandafter\def\csname PY@tok@k\endcsname{\let\PY@bf=\textbf\def\PY@tc##1{\textcolor[rgb]{0.00,0.50,0.00}{##1}}}
\expandafter\def\csname PY@tok@kp\endcsname{\def\PY@tc##1{\textcolor[rgb]{0.00,0.50,0.00}{##1}}}
\expandafter\def\csname PY@tok@kt\endcsname{\def\PY@tc##1{\textcolor[rgb]{0.69,0.00,0.25}{##1}}}
\expandafter\def\csname PY@tok@o\endcsname{\def\PY@tc##1{\textcolor[rgb]{0.40,0.40,0.40}{##1}}}
\expandafter\def\csname PY@tok@ow\endcsname{\let\PY@bf=\textbf\def\PY@tc##1{\textcolor[rgb]{0.67,0.13,1.00}{##1}}}
\expandafter\def\csname PY@tok@nb\endcsname{\def\PY@tc##1{\textcolor[rgb]{0.00,0.50,0.00}{##1}}}
\expandafter\def\csname PY@tok@nf\endcsname{\def\PY@tc##1{\textcolor[rgb]{0.00,0.00,1.00}{##1}}}
\expandafter\def\csname PY@tok@nc\endcsname{\let\PY@bf=\textbf\def\PY@tc##1{\textcolor[rgb]{0.00,0.00,1.00}{##1}}}
\expandafter\def\csname PY@tok@nn\endcsname{\let\PY@bf=\textbf\def\PY@tc##1{\textcolor[rgb]{0.00,0.00,1.00}{##1}}}
\expandafter\def\csname PY@tok@ne\endcsname{\let\PY@bf=\textbf\def\PY@tc##1{\textcolor[rgb]{0.82,0.25,0.23}{##1}}}
\expandafter\def\csname PY@tok@nv\endcsname{\def\PY@tc##1{\textcolor[rgb]{0.10,0.09,0.49}{##1}}}
\expandafter\def\csname PY@tok@no\endcsname{\def\PY@tc##1{\textcolor[rgb]{0.53,0.00,0.00}{##1}}}
\expandafter\def\csname PY@tok@nl\endcsname{\def\PY@tc##1{\textcolor[rgb]{0.63,0.63,0.00}{##1}}}
\expandafter\def\csname PY@tok@ni\endcsname{\let\PY@bf=\textbf\def\PY@tc##1{\textcolor[rgb]{0.60,0.60,0.60}{##1}}}
\expandafter\def\csname PY@tok@na\endcsname{\def\PY@tc##1{\textcolor[rgb]{0.49,0.56,0.16}{##1}}}
\expandafter\def\csname PY@tok@nt\endcsname{\let\PY@bf=\textbf\def\PY@tc##1{\textcolor[rgb]{0.00,0.50,0.00}{##1}}}
\expandafter\def\csname PY@tok@nd\endcsname{\def\PY@tc##1{\textcolor[rgb]{0.67,0.13,1.00}{##1}}}
\expandafter\def\csname PY@tok@s\endcsname{\def\PY@tc##1{\textcolor[rgb]{0.73,0.13,0.13}{##1}}}
\expandafter\def\csname PY@tok@sd\endcsname{\let\PY@it=\textit\def\PY@tc##1{\textcolor[rgb]{0.73,0.13,0.13}{##1}}}
\expandafter\def\csname PY@tok@si\endcsname{\let\PY@bf=\textbf\def\PY@tc##1{\textcolor[rgb]{0.73,0.40,0.53}{##1}}}
\expandafter\def\csname PY@tok@se\endcsname{\let\PY@bf=\textbf\def\PY@tc##1{\textcolor[rgb]{0.73,0.40,0.13}{##1}}}
\expandafter\def\csname PY@tok@sr\endcsname{\def\PY@tc##1{\textcolor[rgb]{0.73,0.40,0.53}{##1}}}
\expandafter\def\csname PY@tok@ss\endcsname{\def\PY@tc##1{\textcolor[rgb]{0.10,0.09,0.49}{##1}}}
\expandafter\def\csname PY@tok@sx\endcsname{\def\PY@tc##1{\textcolor[rgb]{0.00,0.50,0.00}{##1}}}
\expandafter\def\csname PY@tok@m\endcsname{\def\PY@tc##1{\textcolor[rgb]{0.40,0.40,0.40}{##1}}}
\expandafter\def\csname PY@tok@gh\endcsname{\let\PY@bf=\textbf\def\PY@tc##1{\textcolor[rgb]{0.00,0.00,0.50}{##1}}}
\expandafter\def\csname PY@tok@gu\endcsname{\let\PY@bf=\textbf\def\PY@tc##1{\textcolor[rgb]{0.50,0.00,0.50}{##1}}}
\expandafter\def\csname PY@tok@gd\endcsname{\def\PY@tc##1{\textcolor[rgb]{0.63,0.00,0.00}{##1}}}
\expandafter\def\csname PY@tok@gi\endcsname{\def\PY@tc##1{\textcolor[rgb]{0.00,0.63,0.00}{##1}}}
\expandafter\def\csname PY@tok@gr\endcsname{\def\PY@tc##1{\textcolor[rgb]{1.00,0.00,0.00}{##1}}}
\expandafter\def\csname PY@tok@ge\endcsname{\let\PY@it=\textit}
\expandafter\def\csname PY@tok@gs\endcsname{\let\PY@bf=\textbf}
\expandafter\def\csname PY@tok@gp\endcsname{\let\PY@bf=\textbf\def\PY@tc##1{\textcolor[rgb]{0.00,0.00,0.50}{##1}}}
\expandafter\def\csname PY@tok@go\endcsname{\def\PY@tc##1{\textcolor[rgb]{0.53,0.53,0.53}{##1}}}
\expandafter\def\csname PY@tok@gt\endcsname{\def\PY@tc##1{\textcolor[rgb]{0.00,0.27,0.87}{##1}}}
\expandafter\def\csname PY@tok@err\endcsname{\def\PY@bc##1{\setlength{\fboxsep}{0pt}\fcolorbox[rgb]{1.00,0.00,0.00}{1,1,1}{\strut ##1}}}
\expandafter\def\csname PY@tok@kc\endcsname{\let\PY@bf=\textbf\def\PY@tc##1{\textcolor[rgb]{0.00,0.50,0.00}{##1}}}
\expandafter\def\csname PY@tok@kd\endcsname{\let\PY@bf=\textbf\def\PY@tc##1{\textcolor[rgb]{0.00,0.50,0.00}{##1}}}
\expandafter\def\csname PY@tok@kn\endcsname{\let\PY@bf=\textbf\def\PY@tc##1{\textcolor[rgb]{0.00,0.50,0.00}{##1}}}
\expandafter\def\csname PY@tok@kr\endcsname{\let\PY@bf=\textbf\def\PY@tc##1{\textcolor[rgb]{0.00,0.50,0.00}{##1}}}
\expandafter\def\csname PY@tok@bp\endcsname{\def\PY@tc##1{\textcolor[rgb]{0.00,0.50,0.00}{##1}}}
\expandafter\def\csname PY@tok@fm\endcsname{\def\PY@tc##1{\textcolor[rgb]{0.00,0.00,1.00}{##1}}}
\expandafter\def\csname PY@tok@vc\endcsname{\def\PY@tc##1{\textcolor[rgb]{0.10,0.09,0.49}{##1}}}
\expandafter\def\csname PY@tok@vg\endcsname{\def\PY@tc##1{\textcolor[rgb]{0.10,0.09,0.49}{##1}}}
\expandafter\def\csname PY@tok@vi\endcsname{\def\PY@tc##1{\textcolor[rgb]{0.10,0.09,0.49}{##1}}}
\expandafter\def\csname PY@tok@vm\endcsname{\def\PY@tc##1{\textcolor[rgb]{0.10,0.09,0.49}{##1}}}
\expandafter\def\csname PY@tok@sa\endcsname{\def\PY@tc##1{\textcolor[rgb]{0.73,0.13,0.13}{##1}}}
\expandafter\def\csname PY@tok@sb\endcsname{\def\PY@tc##1{\textcolor[rgb]{0.73,0.13,0.13}{##1}}}
\expandafter\def\csname PY@tok@sc\endcsname{\def\PY@tc##1{\textcolor[rgb]{0.73,0.13,0.13}{##1}}}
\expandafter\def\csname PY@tok@dl\endcsname{\def\PY@tc##1{\textcolor[rgb]{0.73,0.13,0.13}{##1}}}
\expandafter\def\csname PY@tok@s2\endcsname{\def\PY@tc##1{\textcolor[rgb]{0.73,0.13,0.13}{##1}}}
\expandafter\def\csname PY@tok@sh\endcsname{\def\PY@tc##1{\textcolor[rgb]{0.73,0.13,0.13}{##1}}}
\expandafter\def\csname PY@tok@s1\endcsname{\def\PY@tc##1{\textcolor[rgb]{0.73,0.13,0.13}{##1}}}
\expandafter\def\csname PY@tok@mb\endcsname{\def\PY@tc##1{\textcolor[rgb]{0.40,0.40,0.40}{##1}}}
\expandafter\def\csname PY@tok@mf\endcsname{\def\PY@tc##1{\textcolor[rgb]{0.40,0.40,0.40}{##1}}}
\expandafter\def\csname PY@tok@mh\endcsname{\def\PY@tc##1{\textcolor[rgb]{0.40,0.40,0.40}{##1}}}
\expandafter\def\csname PY@tok@mi\endcsname{\def\PY@tc##1{\textcolor[rgb]{0.40,0.40,0.40}{##1}}}
\expandafter\def\csname PY@tok@il\endcsname{\def\PY@tc##1{\textcolor[rgb]{0.40,0.40,0.40}{##1}}}
\expandafter\def\csname PY@tok@mo\endcsname{\def\PY@tc##1{\textcolor[rgb]{0.40,0.40,0.40}{##1}}}
\expandafter\def\csname PY@tok@ch\endcsname{\let\PY@it=\textit\def\PY@tc##1{\textcolor[rgb]{0.25,0.50,0.50}{##1}}}
\expandafter\def\csname PY@tok@cm\endcsname{\let\PY@it=\textit\def\PY@tc##1{\textcolor[rgb]{0.25,0.50,0.50}{##1}}}
\expandafter\def\csname PY@tok@cpf\endcsname{\let\PY@it=\textit\def\PY@tc##1{\textcolor[rgb]{0.25,0.50,0.50}{##1}}}
\expandafter\def\csname PY@tok@c1\endcsname{\let\PY@it=\textit\def\PY@tc##1{\textcolor[rgb]{0.25,0.50,0.50}{##1}}}
\expandafter\def\csname PY@tok@cs\endcsname{\let\PY@it=\textit\def\PY@tc##1{\textcolor[rgb]{0.25,0.50,0.50}{##1}}}

\def\PYZbs{\char`\\}
\def\PYZus{\char`\_}
\def\PYZob{\char`\{}
\def\PYZcb{\char`\}}
\def\PYZca{\char`\^}
\def\PYZam{\char`\&}
\def\PYZlt{\char`\<}
\def\PYZgt{\char`\>}
\def\PYZsh{\char`\#}
\def\PYZpc{\char`\%}
\def\PYZdl{\char`\$}
\def\PYZhy{\char`\-}
\def\PYZsq{\char`\'}
\def\PYZdq{\char`\"}
\def\PYZti{\char`\~}
% for compatibility with earlier versions
\def\PYZat{@}
\def\PYZlb{[}
\def\PYZrb{]}
\makeatother


    % Exact colors from NB
    \definecolor{incolor}{rgb}{0.0, 0.0, 0.5}
    \definecolor{outcolor}{rgb}{0.545, 0.0, 0.0}



    
    % Prevent overflowing lines due to hard-to-break entities
    \sloppy 
    % Setup hyperref package
    \hypersetup{
      breaklinks=true,  % so long urls are correctly broken across lines
      colorlinks=true,
      urlcolor=urlcolor,
      linkcolor=linkcolor,
      citecolor=citecolor,
      }
    % Slightly bigger margins than the latex defaults
    
    \geometry{verbose,tmargin=1in,bmargin=1in,lmargin=1in,rmargin=1in}
    
    

    \begin{document}
    
    
    \maketitle
    
    

    
    \begin{Verbatim}[commandchars=\\\{\}]
{\color{incolor}In [{\color{incolor}1}]:} \PY{k+kn}{import} \PY{n+nn}{pandas} \PY{k}{as} \PY{n+nn}{pd}
        \PY{k+kn}{import} \PY{n+nn}{numpy} \PY{k}{as} \PY{n+nn}{np}
        \PY{k+kn}{import} \PY{n+nn}{scipy}\PY{n+nn}{.}\PY{n+nn}{linalg} \PY{k}{as} \PY{n+nn}{la}
        \PY{k+kn}{import} \PY{n+nn}{statsmodels}\PY{n+nn}{.}\PY{n+nn}{api} \PY{k}{as} \PY{n+nn}{sm}
        \PY{k+kn}{from} \PY{n+nn}{sklearn} \PY{k}{import} \PY{n}{linear\PYZus{}model}\PY{p}{,} \PY{n}{model\PYZus{}selection}\PY{p}{,} \PY{n}{metrics}
        \PY{k+kn}{import} \PY{n+nn}{sklearn}
        \PY{k+kn}{import} \PY{n+nn}{plotly}\PY{n+nn}{.}\PY{n+nn}{graph\PYZus{}objs} \PY{k}{as} \PY{n+nn}{go}
        \PY{k+kn}{import} \PY{n+nn}{matplotlib}\PY{n+nn}{.}\PY{n+nn}{pyplot} \PY{k}{as} \PY{n+nn}{plt}
        \PY{k+kn}{import} \PY{n+nn}{pprint}
\end{Verbatim}

    \begin{Verbatim}[commandchars=\\\{\}]
{\color{incolor}In [{\color{incolor}21}]:} \PY{k+kn}{import} \PY{n+nn}{warnings}
         \PY{n}{warnings}\PY{o}{.}\PY{n}{filterwarnings}\PY{p}{(}\PY{l+s+s1}{\PYZsq{}}\PY{l+s+s1}{ignore}\PY{l+s+s1}{\PYZsq{}}\PY{p}{)}
\end{Verbatim}

    \hypertarget{introduction}{%
\section{Introduction}\label{introduction}}

The united states criminal justice system is large complicated machine
that seeks to deliver justice when an offense has been committed. This
system has been slowly evolving as our society and culture have changed.
Many things that Americans take a natural in our criminal justice
systems are quite abnormal among justice systems world wide. Since the
1990s we America has seen a drastic increase in the incarcerated
population. Many Americans believe, as some politicians would have it,
that this drastic increase in incarceration is a reaction to increased
rates of crime, and that this heightened rate is natural and just. To
many it is unclear who is most affected by this drastic change in the
application of justice in America. It is also unclear how they are so
affected.

In this project I am interested in understanding more about the criminal
justice system and the ways in which the law is being applied to
different groups in America. With this particular data set I will be
examining the effects of different factors on sentence length. The
factors I will be examining are: Offense, age, race, gender, and
admission date. There are many things that contribute to sentence
length, however the scope of this project is limited to these factors.

    \hypertarget{data}{%
\section{Data}\label{data}}

\hypertarget{source-and-credibility}{%
\subsection{Source and Credibility}\label{source-and-credibility}}

The data that I will be using in this analysis is gathered from
primarily two sources. The first is the Bureau of Justice Statistics and
the second is a link to a
\href{https://catalog.data.gov/dataset/sentenced-inmates-in-correctional-facilities}{database}
hosted on \href{https://www.data.gov}{Data.gov} and maintained by the
State of Connecticut Department of Corrections. These are highly
credible sources because they are primary sources for the data. These
organizations are official government agencies which collect, maintain,
and report on this data.

\hypertarget{gathering-and-cleaning}{%
\subsection{Gathering and Cleaning}\label{gathering-and-cleaning}}

All of the data which I am using in this report are freely available to
the public. Collection and cleaning was relatively simple as the source
data was well maintained. The data the I collected from the Bureau of
Justice Statistics (BJS) need to be formatted in a way that is easily
read by the Python packages I will be using. This data was prepared in
.xlsx files as to be easily human readable, however this is not
generally easily ingested by programs. I extracted data that I found to
be relevant into separate .csv files and kept the original files for
reference. The files are

\begin{verbatim}
incarceration_counts.csv
jail_population.csv
jail_trends.csv
state_jail_data.csv
incarceration_by_race.csv.
\end{verbatim}

The file that I obtained from the Connecticut Department of Corrections
is a very well maintained database. The largest issue I had with this
file was mild inconsistency with the way in which certain data was
encoded. This was the data that I spend the most time working to
engineer as it is the data set that I intend to use for different
regression-related analyses. This file is

\begin{verbatim}
Sentenced_Inmates_in_Correctional_Facilities.csv.
\end{verbatim}

    \begin{Verbatim}[commandchars=\\\{\}]
{\color{incolor}In [{\color{incolor}2}]:} \PY{n}{inmates} \PY{o}{=} \PY{n}{pd}\PY{o}{.}\PY{n}{read\PYZus{}csv}\PY{p}{(}\PY{l+s+s1}{\PYZsq{}}\PY{l+s+s1}{Sentenced\PYZus{}Inmates\PYZus{}in\PYZus{}Correctional\PYZus{}Facilities.csv}\PY{l+s+s1}{\PYZsq{}}\PY{p}{)}
\end{Verbatim}

    Here are the columns of the data obtained from the Connecticut
Department of Corrections

    \begin{Verbatim}[commandchars=\\\{\}]
{\color{incolor}In [{\color{incolor}3}]:} \PY{n}{pp} \PY{o}{=} \PY{n}{pprint}\PY{o}{.}\PY{n}{PrettyPrinter}\PY{p}{(}\PY{n}{indent}\PY{o}{=}\PY{l+m+mi}{4}\PY{p}{,} \PY{n}{width}\PY{o}{=}\PY{l+m+mi}{80}\PY{p}{)}
        \PY{n}{pp}\PY{o}{.}\PY{n}{pprint}\PY{p}{(}\PY{n}{inmates}\PY{o}{.}\PY{n}{columns}\PY{o}{.}\PY{n}{values}\PY{p}{)}
\end{Verbatim}

    \begin{Verbatim}[commandchars=\\\{\}]
array(['DOWNLOAD DATE', 'IDENTIFIER', 'LATEST ADMISSION DATE', 'RACE',
       'GENDER', 'AGE', 'END SENTENCE DATE', 'OFFENSE', 'FACILITY',
       'DETAINER', 'SENTENCE DAYS',
       'SPECIAL PAROLE END DATE                                                        '],
      dtype=object)

    \end{Verbatim}

    I also created an engineered version that has each of the offences one
hot encoded

    \begin{Verbatim}[commandchars=\\\{\}]
{\color{incolor}In [{\color{incolor}4}]:} \PY{c+c1}{\PYZsh{} reg\PYZus{}df = pd.read\PYZus{}csv(\PYZsq{}regression\PYZus{}df.csv\PYZsq{})}
        \PY{c+c1}{\PYZsh{} pp.pprint(reg\PYZus{}df.columns.values)}
\end{Verbatim}

    These are data sets that I collected from the BJS. I'll show some of the
columns of these data sets

    \begin{Verbatim}[commandchars=\\\{\}]
{\color{incolor}In [{\color{incolor}5}]:} \PY{n}{incar} \PY{o}{=} \PY{n}{pd}\PY{o}{.}\PY{n}{read\PYZus{}csv}\PY{p}{(}\PY{l+s+s1}{\PYZsq{}}\PY{l+s+s1}{incarceration\PYZus{}trends.csv}\PY{l+s+s1}{\PYZsq{}}\PY{p}{)}
        \PY{n}{pop} \PY{o}{=} \PY{n}{pd}\PY{o}{.}\PY{n}{read\PYZus{}csv}\PY{p}{(}\PY{l+s+s1}{\PYZsq{}}\PY{l+s+s1}{jail\PYZus{}population.csv}\PY{l+s+s1}{\PYZsq{}}\PY{p}{)}
        \PY{n}{trend} \PY{o}{=} \PY{n}{pd}\PY{o}{.}\PY{n}{read\PYZus{}csv}\PY{p}{(}\PY{l+s+s1}{\PYZsq{}}\PY{l+s+s1}{jail\PYZus{}trends.csv}\PY{l+s+s1}{\PYZsq{}}\PY{p}{)}
        \PY{n}{state} \PY{o}{=} \PY{n}{pd}\PY{o}{.}\PY{n}{read\PYZus{}csv}\PY{p}{(}\PY{l+s+s1}{\PYZsq{}}\PY{l+s+s1}{state\PYZus{}jail\PYZus{}data.csv}\PY{l+s+s1}{\PYZsq{}}\PY{p}{)}
        \PY{n}{race} \PY{o}{=} \PY{n}{pd}\PY{o}{.}\PY{n}{read\PYZus{}csv}\PY{p}{(}\PY{l+s+s1}{\PYZsq{}}\PY{l+s+s1}{incarceration\PYZus{}by\PYZus{}race.csv}\PY{l+s+s1}{\PYZsq{}}\PY{p}{)}
\end{Verbatim}

    \begin{Verbatim}[commandchars=\\\{\}]
{\color{incolor}In [{\color{incolor}6}]:} \PY{n}{pp}\PY{o}{.}\PY{n}{pprint}\PY{p}{(}\PY{n}{incar}\PY{o}{.}\PY{n}{columns}\PY{o}{.}\PY{n}{values}\PY{p}{)}
        \PY{n+nb}{print}\PY{p}{(}\PY{l+s+s1}{\PYZsq{}}\PY{l+s+se}{\PYZbs{}n}\PY{l+s+s1}{\PYZsq{}}\PY{p}{)}
        \PY{n}{pp}\PY{o}{.}\PY{n}{pprint}\PY{p}{(}\PY{n}{pop}\PY{o}{.}\PY{n}{columns}\PY{o}{.}\PY{n}{values}\PY{p}{)}
        \PY{n+nb}{print}\PY{p}{(}\PY{l+s+s1}{\PYZsq{}}\PY{l+s+se}{\PYZbs{}n}\PY{l+s+s1}{\PYZsq{}}\PY{p}{)}
        \PY{n}{pp}\PY{o}{.}\PY{n}{pprint}\PY{p}{(}\PY{n}{trend}\PY{o}{.}\PY{n}{columns}\PY{o}{.}\PY{n}{values}\PY{p}{)}
        \PY{n+nb}{print}\PY{p}{(}\PY{l+s+s1}{\PYZsq{}}\PY{l+s+se}{\PYZbs{}n}\PY{l+s+s1}{\PYZsq{}}\PY{p}{)}
        \PY{n}{pp}\PY{o}{.}\PY{n}{pprint}\PY{p}{(}\PY{n}{state}\PY{o}{.}\PY{n}{columns}\PY{o}{.}\PY{n}{values}\PY{p}{)}
\end{Verbatim}

    \begin{Verbatim}[commandchars=\\\{\}]
array(['Unnamed: 0', 'Year', 'State prisons', 'Federal prisons',
       'Local jails'], dtype=object)


array(['Unnamed: 0', 'Pre-trial (unadjusted)', 'Convicted (unadjusted)',
       'Held for state prisons', 'Held for immigration authorities',
       'Held for Bureau of Prisons or U.S. Marshals Service',
       'Total held for other authorities', 'Pre-trial (adjusted)',
       'Convicted (adjusted)', 'year'], dtype=object)


array(['State  Jail incarceration rate (2013)', 'Jail growth (1983-2013)',
       'Percent pre-trial (2013)',
       'Percent held for all state and federal authorities (2013)',
       'Percent held for state prisons (2013)',
       'Percent held for immigration authorities (2013)',
       'Percent held for U.S. Marshals Service (2013)',
       'Percent held for other agencies (2013)  ', 'Unnamed: 8'],
      dtype=object)


array(['Unnamed: 0', 'CONFPOP', 'MALE', 'MALE\_PERC', 'JUVMALE',
       'JUVMALE\_PERC', 'FEM', 'FEM\_PERC', 'JUVFEM', 'JUVFEM\_PERC',
       'WHITE', 'WHITE\_PERC', 'BLACK', 'BLACK\_PERC', 'HISP', 'HISP\_PERC',
       'ASIAN', 'ASIAN\_PERC', 'ICE', 'ICE\_PERC', 'BIA'], dtype=object)

    \end{Verbatim}

    \begin{Verbatim}[commandchars=\\\{\}]
{\color{incolor}In [{\color{incolor}66}]:} \PY{n}{incar}\PY{p}{[}\PY{l+s+s1}{\PYZsq{}}\PY{l+s+s1}{State prisons}\PY{l+s+s1}{\PYZsq{}}\PY{p}{]}\PY{o}{.}\PY{n}{values}
         \PY{c+c1}{\PYZsh{} \PYZsh{} incar.rename(columns=\PYZob{}\PYZsq{}Unnamed: 0.1\PYZsq{}:\PYZsq{}Year\PYZsq{}\PYZcb{}, inplace=True)}
         \PY{c+c1}{\PYZsh{} \PYZsh{} incar.to\PYZus{}csv(\PYZsq{}incarceration\PYZus{}trends.csv\PYZsq{})}
         \PY{c+c1}{\PYZsh{} \PYZsh{} incar.drop(\PYZsq{}Unnamed: 0\PYZsq{}, axis=1, inplace=True)}
         \PY{c+c1}{\PYZsh{} incar.to\PYZus{}csv(\PYZsq{}incarceration\PYZus{}trends.csv\PYZsq{})}
         \PY{n}{incar}
\end{Verbatim}

\begin{Verbatim}[commandchars=\\\{\}]
{\color{outcolor}Out[{\color{outcolor}66}]:}     Unnamed: 0  Year  State prisons Federal prisons  Local jails
         0            0  1925        85239.0           6,430          NaN
         1            1  1926        91188.0          6803.0          NaN
         2            2  1927       101624.0          7722.0          NaN
         3            3  1928       108157.0          8233.0          NaN
         4            4  1929       107532.0         12964.0          NaN
         ..         {\ldots}   {\ldots}            {\ldots}             {\ldots}          {\ldots}
         87          87  2012      1315856.0          196574     744524.0
         88          88  2013      1325305.0          195098     731208.0
         89          89  2014      1316407.0          191374     744592.0
         90          90  2015      1298159.0          178688     727400.0
         91          91  2016      1286691.0          171482     740700.0
         
         [92 rows x 5 columns]
\end{Verbatim}
            
    \hypertarget{visualization}{%
\section{Visualization}\label{visualization}}

    \begin{Verbatim}[commandchars=\\\{\}]
{\color{incolor}In [{\color{incolor}27}]:} \PY{n}{fig} \PY{o}{=} \PY{n}{go}\PY{o}{.}\PY{n}{Figure}\PY{p}{(}
             \PY{n}{layout}\PY{o}{=}\PY{p}{\PYZob{}}\PY{l+s+s1}{\PYZsq{}}\PY{l+s+s1}{title}\PY{l+s+s1}{\PYZsq{}}\PY{p}{:}\PY{p}{\PYZob{}}\PY{l+s+s1}{\PYZsq{}}\PY{l+s+s1}{text}\PY{l+s+s1}{\PYZsq{}}\PY{p}{:}\PY{l+s+s1}{\PYZsq{}}\PY{l+s+s1}{incarceration trends over time}\PY{l+s+s1}{\PYZsq{}}\PY{p}{\PYZcb{}}\PY{p}{,}
                    \PY{l+s+s1}{\PYZsq{}}\PY{l+s+s1}{showlegend}\PY{l+s+s1}{\PYZsq{}}\PY{p}{:}\PY{k+kc}{True}\PY{p}{\PYZcb{}}
         \PY{p}{)}
         \PY{n}{fig}\PY{o}{.}\PY{n}{add\PYZus{}trace}\PY{p}{(}
             \PY{n}{go}\PY{o}{.}\PY{n}{Scatter}\PY{p}{(}
                 \PY{n}{x}\PY{o}{=}\PY{n}{incar}\PY{p}{[}\PY{l+s+s1}{\PYZsq{}}\PY{l+s+s1}{Year}\PY{l+s+s1}{\PYZsq{}}\PY{p}{]}\PY{o}{.}\PY{n}{values}\PY{p}{,} \PY{n}{y}\PY{o}{=}\PY{n}{incar}\PY{p}{[}\PY{l+s+s1}{\PYZsq{}}\PY{l+s+s1}{State prisons}\PY{l+s+s1}{\PYZsq{}}\PY{p}{]}\PY{o}{.}\PY{n}{values}\PY{p}{,}
                 \PY{n}{name}\PY{o}{=}\PY{l+s+s1}{\PYZsq{}}\PY{l+s+s1}{State Prisons}\PY{l+s+s1}{\PYZsq{}}
             \PY{p}{)}
         \PY{p}{)}
         \PY{n}{fig}\PY{o}{.}\PY{n}{add\PYZus{}trace}\PY{p}{(}
             \PY{n}{go}\PY{o}{.}\PY{n}{Scatter}\PY{p}{(}
                 \PY{n}{x}\PY{o}{=}\PY{n}{incar}\PY{p}{[}\PY{l+s+s1}{\PYZsq{}}\PY{l+s+s1}{Year}\PY{l+s+s1}{\PYZsq{}}\PY{p}{]}\PY{o}{.}\PY{n}{values}\PY{p}{,} \PY{n}{y}\PY{o}{=}\PY{n}{incar}\PY{p}{[}\PY{l+s+s1}{\PYZsq{}}\PY{l+s+s1}{Federal prisons}\PY{l+s+s1}{\PYZsq{}}\PY{p}{]}\PY{o}{.}\PY{n}{values}\PY{p}{,}
                 \PY{n}{name}\PY{o}{=}\PY{l+s+s1}{\PYZsq{}}\PY{l+s+s1}{Federal Prisons}\PY{l+s+s1}{\PYZsq{}}
             \PY{p}{)}
         \PY{p}{)}
         \PY{n}{fig}\PY{o}{.}\PY{n}{add\PYZus{}trace}\PY{p}{(}
             \PY{n}{go}\PY{o}{.}\PY{n}{Scatter}\PY{p}{(}
                 \PY{n}{x}\PY{o}{=}\PY{n}{incar}\PY{p}{[}\PY{l+s+s1}{\PYZsq{}}\PY{l+s+s1}{Year}\PY{l+s+s1}{\PYZsq{}}\PY{p}{]}\PY{o}{.}\PY{n}{values}\PY{p}{,} \PY{n}{y}\PY{o}{=}\PY{n}{incar}\PY{p}{[}\PY{l+s+s1}{\PYZsq{}}\PY{l+s+s1}{Local jails}\PY{l+s+s1}{\PYZsq{}}\PY{p}{]}\PY{o}{.}\PY{n}{values}\PY{p}{,}
                 \PY{n}{name}\PY{o}{=}\PY{l+s+s1}{\PYZsq{}}\PY{l+s+s1}{Local Jails}\PY{l+s+s1}{\PYZsq{}}
             \PY{p}{)}
         \PY{p}{)}
         \PY{n}{fig}\PY{o}{.}\PY{n}{show}\PY{p}{(}\PY{p}{)}
\end{Verbatim}

    
    \begin{center}
    \image{newplot.png}
    \end{center}
    { \hspace*{\fill} \\}
    
    
    \hypertarget{increased-rates}{%
\subsection{Increased rates}\label{increased-rates}}

Here we can see the drastic increase of the rate and amount of
incarceration in the U.S. Something that is interesting to note is that
Jails are defined as places for people who have a sentence less than 1
year, or who are awaiting trial. So we see that at its peak in 2008,
there were more people awaiting trial than there were being held in
federal prison in 1991.

    \begin{Verbatim}[commandchars=\\\{\}]
{\color{incolor}In [{\color{incolor}87}]:} \PY{n}{fig}\PY{p}{,} \PY{n}{ax} \PY{o}{=} \PY{n}{plt}\PY{o}{.}\PY{n}{subplots}\PY{p}{(}\PY{l+m+mi}{1}\PY{p}{)}
         \PY{n}{ax}\PY{o}{.}\PY{n}{set\PYZus{}xlim}\PY{p}{(}\PY{o}{\PYZhy{}}\PY{l+m+mi}{50}\PY{p}{,}\PY{l+m+mi}{20000}\PY{p}{)}
         \PY{n}{bp} \PY{o}{=} \PY{n}{inmates}\PY{o}{.}\PY{n}{boxplot}\PY{p}{(}
             \PY{p}{[}\PY{l+s+s1}{\PYZsq{}}\PY{l+s+s1}{SENTENCE DAYS}\PY{l+s+s1}{\PYZsq{}}\PY{p}{]}\PY{p}{,} \PY{n}{by}\PY{o}{=}\PY{p}{[}\PY{l+s+s1}{\PYZsq{}}\PY{l+s+s1}{RACE}\PY{l+s+s1}{\PYZsq{}}\PY{p}{]}\PY{p}{,}
             \PY{n}{vert}\PY{o}{=}\PY{k+kc}{False}\PY{p}{,} \PY{n}{grid}\PY{o}{=}\PY{k+kc}{False}\PY{p}{,}
             \PY{n}{ax}\PY{o}{=}\PY{n}{ax}
         \PY{p}{)}
         \PY{n}{plt}\PY{o}{.}\PY{n}{show}\PY{p}{(}\PY{p}{)}
\end{Verbatim}

    \begin{center}
    \adjustimage{max size={0.9\linewidth}{0.9\paperheight}}{output_16_0.png}
    \end{center}
    { \hspace*{\fill} \\}
    
    \hypertarget{racial-disparity}{%
\subsection{Racial Disparity}\label{racial-disparity}}

Here we can see some preliminary findings related to racial disparities
in the criminal justice system. Though we do see comparable median
sentence lengths, the variance in sentence length among black inmates is
much higher and the spread of the top two quintiles is much wider.

    \begin{Verbatim}[commandchars=\\\{\}]
{\color{incolor}In [{\color{incolor}26}]:} \PY{c+c1}{\PYZsh{} fig = go.Figure(}
         \PY{c+c1}{\PYZsh{}     layout=\PYZob{}\PYZsq{}title\PYZsq{}:\PYZob{}\PYZsq{}text\PYZsq{}:\PYZsq{}Comparing Age and Sentence Lenght\PYZsq{}\PYZcb{},}
         \PY{c+c1}{\PYZsh{}            \PYZsq{}showlegend\PYZsq{}:False\PYZcb{}}
         \PY{c+c1}{\PYZsh{} )}
         \PY{c+c1}{\PYZsh{} fig.add\PYZus{}trace(}
         \PY{c+c1}{\PYZsh{}     go.Scatter(}
         \PY{c+c1}{\PYZsh{}         x=inmates[\PYZsq{}AGE\PYZsq{}].values, y=inmates[\PYZsq{}SENTENCE DAYS\PYZsq{}].values}
         \PY{c+c1}{\PYZsh{}     )}
         \PY{c+c1}{\PYZsh{} )}
         \PY{c+c1}{\PYZsh{} fig.show()}
         \PY{n}{fig}\PY{p}{,} \PY{n}{ax} \PY{o}{=} \PY{n}{plt}\PY{o}{.}\PY{n}{subplots}\PY{p}{(}\PY{l+m+mi}{1}\PY{p}{)}
         \PY{n}{ax}\PY{o}{.}\PY{n}{set\PYZus{}ylim}\PY{p}{(}\PY{l+m+mi}{0}\PY{p}{,}\PY{l+m+mi}{10000}\PY{p}{)}
         \PY{n}{mask} \PY{o}{=} \PY{n}{inmates}\PY{p}{[}\PY{l+s+s1}{\PYZsq{}}\PY{l+s+s1}{SENTENCE DAYS}\PY{l+s+s1}{\PYZsq{}}\PY{p}{]} \PY{o}{\PYZlt{}} \PY{l+m+mi}{10000}
         \PY{n}{ax}\PY{o}{.}\PY{n}{hexbin}\PY{p}{(}
             \PY{n}{x}\PY{o}{=}\PY{n}{inmates}\PY{p}{[}\PY{l+s+s1}{\PYZsq{}}\PY{l+s+s1}{AGE}\PY{l+s+s1}{\PYZsq{}}\PY{p}{]}\PY{p}{[}\PY{n}{mask}\PY{p}{]}\PY{o}{.}\PY{n}{values}\PY{p}{,}
             \PY{n}{y}\PY{o}{=}\PY{n}{inmates}\PY{p}{[}\PY{l+s+s1}{\PYZsq{}}\PY{l+s+s1}{SENTENCE DAYS}\PY{l+s+s1}{\PYZsq{}}\PY{p}{]}\PY{p}{[}\PY{n}{mask}\PY{p}{]}\PY{o}{.}\PY{n}{values}\PY{p}{,}
             \PY{n}{gridsize}\PY{o}{=}\PY{l+m+mi}{45}
         \PY{p}{)}
         \PY{n}{plt}\PY{o}{.}\PY{n}{show}\PY{p}{(}\PY{p}{)}
\end{Verbatim}

    \begin{center}
    \adjustimage{max size={0.9\linewidth}{0.9\paperheight}}{output_18_0.png}
    \end{center}
    { \hspace*{\fill} \\}
    
    \hypertarget{age-discrepency}{%
\subsection{Age Discrepency}\label{age-discrepency}}

Here we can see that there is slight bias against young people, as they
seem to get slightly longer sentences. Mostly we see that young people
are more likely to be arrested and convicted of a crime than older
people are.

\hypertarget{standard-sentence-length}{%
\subsection{Standard Sentence Length}\label{standard-sentence-length}}

Something interesting that emerges from this hexbin plot is a depiction
of standard sentence lengths. These became more common as the idea of
mandatory minimum sentences has become more popular. This is seen in the
plot as straight bright horizontal lines. Sentences are given at these
standard lengths, often times because it is not legal to give a shorter
sentence.

    \begin{Verbatim}[commandchars=\\\{\}]
{\color{incolor}In [{\color{incolor} }]:} 
\end{Verbatim}


    % Add a bibliography block to the postdoc
    
    
    
    \end{document}
